\documentclass{article}
\usepackage[utf8]{inputenc}
\usepackage[normalem]{ulem}
\usepackage{indentfirst}

\begin{document}

\section{Greedy Set Cover}
For the project, we will be looking at the greedy algorithm for solving general set cover problems. The greedy algorithm is an approximation algorithm with a guaranteed performance compared with the optimal solution to the problem. Unlike most approximation algorithms, the guarantee is not a constant multiple of the optimal but instead a it is a function multiple of the optimal based on the size of the input for the instance. The greedy algorithm is guaranteed provide a solution that is less than or equal to $\alpha \cdot OPT(I)$ where $\alpha$ is given by $$\alpha = H_k = \sum_{i=1}^k \frac{1}{i}\leq 1+\log(k)$$ where $k$ is the size of the largest subset and $H_k$ is the $k$th partial sum of the harmonic series. Since $k$ will always be less than $n$, we can re-frame $\alpha$ in terms of $n$ if so desired.

The implementation of the algorithm is fairly simple. Initialize an array that holds the cost effectiveness for each subset where cost effectiveness is the cost of the subset divided by the number of elements in the subset. Take the set with the lowest cost effectiveness and consider all elements in the subset covered. Update the cost effectiveness of all sets excluding the elements covered by the previous set. Repeat the process until all elements in the set are covered. The subsets chosen through the process are the result of the algorithm. The algorithm works on unweighted set cover where the weights of each subset is simply 1. The algorithm has a worst-case running time of $O(N\log(N))$ where $N$ is the actual size of the input including all subsets.

The guarantee of $H_k\cdot OPT(I)$ seems weak compared to algorithms for other NP-Hard problems where there are 2-approximation guarantees. The harmonic series quickly increases past 2, passing it for $k>4$. However, it has been shown that the efficiency of the greedy algorithm for general set cover problems cannot be improved to a constant nor to anything significantly more efficient than the greedy algorithm (Pusztai, 2008).

\section{References}
Pusztai, Pál. "An Application of the Greedy Heuristic of Set Cover to Traffic Checks." Central European Journal of Operations Research 16.4 (2008): 407-14. Print.

Young, Neal E. "Greedy Set-Cover Algorithms." Encyclopedia of Algorithms. Ed. Ming-Yang Kao. New York, NY: Springer, 2008. 379-81.

\end{document}